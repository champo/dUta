\documentclass[11pt,a4paper,titlepage]{article}
\usepackage[utf8]{inputenc}
\usepackage[spanish]{babel}
\usepackage{amsmath}
\usepackage{amsfonts}
\usepackage{amssymb}
\usepackage{makeidx}
\usepackage{graphicx}

\title{Preliminary Report}
\author{Civile, Juan \and Sneidermanis, Dario \and Kenny, Kevin}
\date{9 de Mayo del 2012}

\begin{document}

\newcommand{\awesome}[1]{\texttt{\large #1 -}}

\maketitle
\tableofcontents
\clearpage

\section{RFCs consultados}

\begin{itemize}

    \item \awesome{RFC 822}  Standard for the format of ARPA Internet text messages
    \item \awesome{RFC 2119} Key words for use in RFCs to Indicate Requirement Levels
    \item \awesome{RFC 2396} Uniform Resource Identifiers (URI): Generic Syntax
    \item \awesome{RFC 1945} Hypertext Transfer Protocol -- HTTP/1.0
    \item \awesome{RFC 2068} Hypertext Transfer Protocol -- HTTP/1.1
    \item \awesome{RFC 2616} Hypertext Transfer Protocol -- HTTP/1.1
    \item \awesome{RFC 2046} Multipurpose Internet Mail Extensions (MIME) Part Two: Media Types
    \item \awesome{RFC 2047} MIME (Multipurpose Internet Mail Extensions) Part Three: Message Header Extensions for Non-ASCII Text
    \item \awesome{RFC 2324} Hyper Text Coffee Pot Control Protocol (HTCPCP/1.0)

\end{itemize}

\section{Diseño de los Protocolos}

\section{Analisis de los RFCs}

\section{Herramientas?}
    Durante el desarrollo vamos a utilizar las siguientes herramientas:
    \begin{itemize}
        \item \awesome{JMeter} Herramienta para benchmarking de aplicaciones
        \item \awesome{Apache Bench} Herramienta de benchmarking para servidores HTTP
        \item \awesome{nginx} Servidor HTTP
        \item \awesome{8tracks} \emph{``Handcrafted internet radio''}
        \item \awesome{curl} Herramienta de transferencia de datos
    \end{itemize}

\section{Casos de prueba}

\end{document}
