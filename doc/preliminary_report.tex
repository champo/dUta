\documentclass[11pt,a4paper,titlepage]{article}
\usepackage[utf8]{inputenc}
\usepackage[spanish]{babel}
\usepackage{amsmath}
\usepackage{amsfonts}
\usepackage{amssymb}
\usepackage{makeidx}
\usepackage{graphicx}

\title{Preliminary Report}
\author{Civile, Juan \and Sneidermanis, Dario \and Kenny, Kevin}
\date{9 de Mayo del 2012}

\begin{document}

\newcommand{\awesome}[1]{\texttt{\large #1}}
\newcommand{\ua}{\textit{User Agent} }
\newcommand{\os}{\textit{Origin Server} }

\maketitle
\tableofcontents
\clearpage

\section{RFCs consultados}

\begin{itemize}

    \item \awesome{RFC 822}  - Standard for the format of ARPA Internet text messages
    \item \awesome{RFC 2119} - Key words for use in RFCs to Indicate Requirement Levels
    \item \awesome{RFC 2396} - Uniform Resource Identifiers (URI): Generic Syntax
    \item \awesome{RFC 1945} - Hypertext Transfer Protocol -- HTTP/1.0
    \item \awesome{RFC 2068} - Hypertext Transfer Protocol -- HTTP/1.1
    \item \awesome{RFC 2616} - Hypertext Transfer Protocol -- HTTP/1.1
    \item \awesome{RFC 2046} - Multipurpose Internet Mail Extensions (MIME) Part Two: Media Types
    \item \awesome{RFC 2047} - MIME (Multipurpose Internet Mail Extensions) Part Three: Message Header Extensions for Non-ASCII Text
    \item \awesome{RFC 2324} - Hyper Text Coffee Pot Control Protocol (HTCPCP/1.0)

\end{itemize}

\section{Diseño de los Protocolos}

\section{Analisis de los RFCs}

\section{Herramientas}
    Durante el desarrollo vamos a utilizar las siguientes herramientas:
    \begin{itemize}
        \item \awesome{JMeter} - Herramienta para benchmarking de aplicaciones
        \item \awesome{Apache Bench} - Herramienta de benchmarking para servidores HTTP
        \item \awesome{nginx} - Servidor HTTP
        \item \awesome{8tracks} - \emph{``Handcrafted internet radio''}
        \item \awesome{curl} - Herramienta de transferencia de datos
        \item \awesome{netcat} - Utilidad para transferencia de datos por TCP y UDP
        \item \awesome{Chromium} - Web browser
    \end{itemize}

\section{Casos de prueba}
% Con netcat para mandar edge cases del formato de los headers
\subsection{Headers con campos multilinea y con multiples valores}
Usando la herramienta \awesome{netcat}, enviaremos requests que en sus headers contengan casos especiales.
Uno de estos casos es tener un campo multilinea.
El otro es tener multiples instancias de un mismo header, como por ejemplo \textit{Cookie}.

% Con curl para mandar metodos no soportados (TRACE)
\subsection{Requests invalidos}
Compraberemos que el proxy responde con el status code adecuado cuando recibe requests con metodos que no soporta.
Por ejemplo, \awesome{TRACE}.

% Contra un server que sirva archivos grandes
\subsection{Mensajes grandes}
% TODO: Write something

\subsection{Throughtput}
Para medir la capacidad del proxy, utilizaremos \awesome{JMeter}.
El objetivo es obtener metricas como latencia y requests por segundo para una variedad de escenarios.
Los escenarios tendran variaciones en: 
\begin{itemize}
    \item Requests totales
    \item Requests concurrentes
    \item Servidor destino
    \item Filtros y transformaciones aplicadas
\end{itemize}

\subsection{Filtros}
Consideramos como \textit{filtros} a las reglas de control de acceso y transformaciones que puede aplicar el proxy.

\subsubsection{Transfer-Encoding}
Ante una respuesta con \textit{Transfer-Encoding: chunked} un filtro debe correctamente reconstruir el mensaje y luego aplicar la logica que corresponda.

\subsubsection{Cache Headers}
Podria ocurrir que el \os devuelva contenido con \textit{Cache headers} al que se aplica un filtro en el proxy.
Luego, el filtro es desactivado y el \ua vuelve a pedir este contenido con un \awesome{GET} condicional.
Si no se manipularon correctamente los headers de la respuesta original, el \os puede retornar un status code 304, haciendo que el \ua muestre contenido invalido.

\subsubsection{Multipart MIMEs}
El mensaje entregado por un \os puede ser del tipo \textit{multipart/*}.
Este tipo de mensajes puede a su vez contener partes a las que apliquen filtros.
Por lo tanto el proxy tiene que ser capaz de procesar estas partes individualmente y luego entregar un mensaje con todas ellas, sin perdida de informacion.

% Da incluir algo sobre archivos grandes? No se como vamos a manejar esto, asi que me pa que no da.

\end{document}
