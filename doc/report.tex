\documentclass[11pt,a4paper,titlepage]{article}
\usepackage[utf8]{inputenc}
\usepackage[spanish]{babel}
\usepackage{amsmath}
\usepackage{amsfonts}
\usepackage{amssymb}
\usepackage{makeidx}
\usepackage{graphicx}
\usepackage{tipa}
\usepackage{ulem}
\usepackage{multicol}


\title{Report - dUta Proxy Server}
\author{Civile, Juan\ \  \and Sneidermanis, Dario \and Kenny, Kevin}
\date{6 de Junio del 2012}

\begin{document}

\newcommand{\awesome}[1]{\texttt{\large #1}}
\newcommand{\ua}{\textit{User Agent} }
\newcommand{\os}{\textit{Origin Server} }
\newcommand{\duta}{\awesome{dUta} }

\maketitle
\tableofcontents
\clearpage

\section{RFCs consultados}
% {{{

\begin{itemize}

    \item \awesome{RFC 822}  - Standard for the format of ARPA Internet text messages
    \item \awesome{RFC 2119} - Key words for use in RFCs to Indicate Requirement Levels
    \item \awesome{RFC 2396} - Uniform Resource Identifiers (URI): Generic Syntax
    \item \awesome{RFC 1945} - Hypertext Transfer Protocol -- HTTP/1.0
    \item \awesome{RFC 2068} - Hypertext Transfer Protocol -- HTTP/1.1
    \item \awesome{RFC 2616} - Hypertext Transfer Protocol -- HTTP/1.1
    \item \awesome{RFC 2046} - Multipurpose Internet Mail Extensions (MIME) Part Two: Media Types
    \item \awesome{RFC 2047} - MIME (Multipurpose Internet Mail Extensions) Part Three: Message Header Extensions for Non-ASCII Text
    \item \sout{\awesome{RFC 2324} - Hyper Text Coffee Pot Control Protocol (HTCPCP/1.0)}
    \item \awesome{RFC 4627} - The application/json Media Type for JavaScript Object Notation (JSON)
    \item \awesome{RFC 2617} - HTTP Authentication: Basic and Digest Access Authentication
\end{itemize}

% }}}

\section{Diseño}
% {{{
% }}}

\section{Protocolo de administracion}
% {{{

Para esta seccion, tomamos la sintaxis definida por el RFC 2616 en las secciones 2.1 ``Augmented BNF'' y 2.2 ``Basic Rules''.
También tomamos las definiciones de las secciones 2.5 ``Numbers'' y 2.6 ``Strings'' del RFC 4627.
En caso de definiciones conflictivas, se toma con mayor precedencia las definiciones del RFC 4627.

El protocolo de configuración y monitoreo de \duta utiliza HTTP 1.1 como transporte.
El proxy presenta un servidor HTTP en un puerto especial, este valor es configurable y por defecto es 1337.
Este servidor esta implementando como un filtro mas del proxy, y por lo tanto, soporta todas las mismas funcionalidades HTTP que el proxy.

Los mensajes que recibe y envía este servidor son del tipo \textit{application/json}, como definido en el RFC 4627.
También se hara uso de la \textit{Basic Authentication Scheme} definida en el RFC 2617.
Cualquier recurso o mensaje que no esté detallado en la siguiente sección, producirá un status code de error correspondiente.

\subsection{Nuevo filtro}
\label{sec:new-filter}
Para configurar nuevos filtros se debe hacer un request \awesome{POST} al recurso \awesome{/filter} con un mensaje del formato \textit{filter}.
De ser agregado correctamente, la respuesta tendrá status code 201 y especificará mediante \textit{Location} el recurso asociado al nuevo filtro.
En caso de encontrar un error con el mensaje enviado, se retornara un status code 400, y de ser posible, un mensaje que indique el error.

\begin{verbatim}
filter =
    ("{ 'type': " type ", 'apply': " apply "}") |
    ("{ 'type': " type ", 'apply': " apply ", 'config': " config "}")

type =
    "'deny-all'" |
    "'deny-ip'" |
    "'deny-url'" |
    "'deny-type'" |
    "'deny-size'" |
    "'l33t'" |
    "'rotate'"

apply = "[" apply-rules "]"

apply-rules =
    apply-rule |
    (apply-rule ", " apply-rules)

apply-rule =
    "{ 'ip': " string "}" |
    "{ 'browser': " browser "}" |
    "{ 'os': " os "}"

\end{verbatim}

\textit{browser} es un string con uno de los siguientes valores:

\begin{multicols}{4}

    \small \texttt{APPLE\_MAIL}

    \small \texttt{BOT}

    \small \texttt{CAMINO}

    \small \texttt{CAMINO2}

    \small \texttt{CFNETWORK}

    \small \texttt{CHROME}

    \small \texttt{CHROME\_MOBILE}

    \small \texttt{CHROME10}

    \small \texttt{CHROME11}

    \small \texttt{CHROME12}

    \small \texttt{CHROME13}

    \small \texttt{CHROME14}

    \small \texttt{CHROME15}

    \small \texttt{CHROME16}

    \small \texttt{CHROME17}

    \small \texttt{CHROME18}

    \small \texttt{CHROME19}

    \small \texttt{CHROME8}

    \small \texttt{CHROME9}

    \small \texttt{DOLFIN2}

    \small \texttt{DOWNLOAD}

    \small \texttt{EUDORA}

    \small \texttt{EVOLUTION}

    \small \texttt{FIREFOX}

    \small \texttt{FIREFOX1\_5}

    \small \texttt{FIREFOX10}

    \small \texttt{FIREFOX11}

    \small \texttt{FIREFOX12}

    \small \texttt{FIREFOX13}

    \small \texttt{FIREFOX2}

    \small \texttt{FIREFOX3}

    \small \texttt{FIREFOX3MOBILE}

    \small \texttt{FIREFOX4}

    \small \texttt{FIREFOX5}

    \small \texttt{FIREFOX6}

    \small \texttt{FIREFOX7}

    \small \texttt{FIREFOX8}

    \small \texttt{FIREFOX9}

    \small \texttt{FLOCK}

    \small \texttt{IE}

    \small \texttt{IE10}

    \small \texttt{IE5}

    \small \texttt{IE5\_5}

    \small \texttt{IE6}

    \small \texttt{IE7}

    \small \texttt{IE8}

    \small \texttt{IE9}

    \small \texttt{IEMOBILE6}

    \small \texttt{IEMOBILE7}

    \small \texttt{IEMOBILE9}

    \small \texttt{KONQUEROR}

    \small \texttt{LOTUS\_NOTES}

    \small \texttt{LYNX}

    \small \texttt{MOBILE\_SAFARI}

    \small \texttt{MOZILLA}

    \small \texttt{NETFRONT}

    \small \texttt{OMNIWEB}

    \small \texttt{OPERA}

    \small \texttt{OPERA\_MINI}

    \small \texttt{OPERA10}

    \small \texttt{OPERA9}

    \small \texttt{OUTLOOK}

    \small \texttt{OUTLOOK\_EXPRESS7}

    \small \texttt{OUTLOOK2007}

    \small \texttt{OUTLOOK2010}

    \small \texttt{POCOMAIL}

    \small \texttt{SAFARI}

    \small \texttt{SAFARI4}

    \small \texttt{SAFARI5}

    \small \texttt{SEAMONKEY}

    \small \texttt{SILK}

    \small \texttt{THEBAT}

    \small \texttt{THUNDERBIRD}

    \small \texttt{THUNDERBIRD10}

    \small \texttt{THUNDERBIRD11}

    \small \texttt{THUNDERBIRD12}

    \small \texttt{THUNDERBIRD2}

    \small \texttt{THUNDERBIRD3}

    \small \texttt{THUNDERBIRD6}

    \small \texttt{THUNDERBIRD7}

    \small \texttt{THUNDERBIRD8}

    \small \texttt{UNKNOWN}

\end{multicols}

\textit{os} es un string con uno de los siguientes valores:

\begin{multicols}{4}

    \small \texttt{ANDROID}

    \small \texttt{ANDROID1}

    \small \texttt{ANDROID2}

    \small \texttt{ANDROID2\_TABLET}

    \small \texttt{ANDROID3\_TABLET}

    \small \texttt{ANDROID4}

    \small \texttt{ANDROID4\_TABLET}

    \small \texttt{BADA}

    \small \texttt{BLACKBERRY}

    \small \texttt{BLACKBERRY\_TABLET}

    \small \texttt{BLACKBERRY6}

    \small \texttt{BLACKBERRY7}

    \small \texttt{GOOGLE\_TV}

    \small \texttt{IOS}

    \small \texttt{iOS4\_IPHONE}

    \small \texttt{iOS5\_IPHONE}

    \small \texttt{KINDLE}

    \small \texttt{KINDLE2}

    \small \texttt{KINDLE3}

    \small \texttt{LINUX}

    \small \texttt{MAC\_OS}

    \small \texttt{MAC\_OS\_X}

    \small \texttt{MAC\_OS\_X\_IPAD}

    \small \texttt{MAC\_OS\_X\_IPHONE}

    \small \texttt{MAC\_OS\_X\_IPOD}

    \small \texttt{MAEMO}

    \small \texttt{PALM}

    \small \texttt{PSP}

    \small \texttt{ROKU}

    \small \texttt{SERIES40}

    \small \texttt{SONY\_ERICSSON}

    \small \texttt{SUN\_OS}

    \small \texttt{SYMBIAN}

    \small \texttt{SYMBIAN6}

    \small \texttt{SYMBIAN7}

    \small \texttt{SYMBIAN8}

    \small \texttt{SYMBIAN9}

    \small \texttt{UNKNOWN}

    \small \texttt{WEBOS}

    \small \texttt{WII}

    \small \texttt{WINDOWS}

    \small \texttt{WINDOWS\_2000}

    \small \texttt{WINDOWS\_7}

    \small \texttt{WINDOWS\_98}

    \small \texttt{WINDOWS\_MOBILE}

    \small \texttt{WINDOWS\_MOBILE7}

    \small \texttt{WINDOWS\_VISTA}

    \small \texttt{WINDOWS\_XP}

\end{multicols}


La definición de \textit{host-string} es mixta.
Debe respetar las reglas de \textit{string} definidas en el RFC 4627 y contener un valor válido según el \textit{host} aceptado por el RFC 2616.

El contenido aceptado por \textit{config} y si debe ser omitido o no, sera determinado por el valor de \textit{type}.
Los valores \textit{deny-all}, \textit{l33t} y \textit{rotate} no deben incluir \textit{config}.
\subsubsection{deny-ip}
\begin{verbatim}
config = host-string
\end{verbatim}

\subsubsection{deny-url}
\begin{verbatim}
config = uri
uri = string
\end{verbatim}

\subsubsection{deny-type}
\begin{verbatim}
config = mime
mime = string
\end{verbatim}

\subsubsection{deny-size}
\begin{verbatim}
config = size
size = int
\end{verbatim}

\subsection{Remover filtros}
Para remover un filtro configurado, se debe enviar un request con método \textit{DELETE} y mensaje vacío al recurso asociado al filtro.

\subsection{Monitoreo}
Para obtener los valores de monitoreo, existirá un recurso por cada posible categoría.
Al enviar un request con método \awesome{GET}, el servidor responderá con el valor correspondiente.
Los mensajes contenidos en las respuestas, todos tendrán el mismo formato, \textit{value}.
\begin{verbatim}
value = "{ 'value': " int "} "
\end{verbatim}

Los recursos disponibles son:
\begin{itemize}
    \item \textit{/stats/bytes}
    \item \textit{/stats/bytes/clients}
    \item \textit{/stats/bytes/servers}
    \item \textit{/stats/filter/type} --- Donde \textit{type} es es el definido en \ref{sec:new-filter}.
    \item \textit{/stats/channels}
    \item \textit{/stats/channels/clients}
    \item \textit{/stats/channels/servers}
\end{itemize}
% }}}

\section{Problemas encontrados}
% {{{
    \subsection{Concurrencia}
    % AKA locking is haaaard

    \subsection{Manejo de mensajes grandes}
    % And so, file storage came to be

    \subsection{Delimitacion de mensajes}
    % On the 3rd day came DataBuffer

    \subsection{Limite de file descriptors}
    % This is relleno.
% }}}

\section{Limitaciones}
% {{{
    \subsection{Multipart MIME types}

    \subsection{Transformaciones de mensajes grandes}

    \subsection{HTTP 1.0}

    \subsection{Retry}

    \subsection{Content-Encoding}

% }}}

\section{Posibles extensiones}
% {{{
    \subsection{Pipelining}

    \subsection{Reducir el footprint de memoria}

    \subsection{Worker threads}
% }}}

\section{Conclusiones}
% {{{
    WTF
% }}}

\section{Casos de prueba}
% {{{
    \subsection{\awesome{Firefox 12}}
    Utilizamos una instalacion de \awesome{Firefox 12} con el plugin \awesome{HttpFox}.
    \awesome{HttpFox} nos permite revisar los pedidos enviados por \awesome{Firefox} y las respuestas dadas por \duta.

    Una vez configurado \awesome{Firefox} para utilizar el proxy, procedimos a cargar:
    \begin{itemize}
        \item http://www.google.com.ar/
        \item http://www.terra.es/
        \item http://www.facebook.com/
        \item http://en.wikipedia.org/wiki/Casa
    \end{itemize}

    \subsection{Bloqueo por URI}

    \subsection{Rotacion de imagenes}

    \subsection{Bloqueo total por browser}

    \subsection{l33t}

    \subsection{Bloqueo por tamaño}

% }}}

\section{Deploy}
% {{{
    \subsection{Build}
    Para el proceso de build utilizamos la herramienta \awesome{Maven}.
    Simplemente ejecutar \texttt{mvn clean install} genera un archivo \textit{.jar} listo para ejecutar \duta.
    El archivo se encontrara bajo la carpeta \textit{target} y tendra el nombre \textit{dUta-1.0-SNAPSHOT-jar-with-dependencies.jar}.

    \subsection{Logs}
    \duta por defecto utiliza 2 logs.
    Uno es \textit{access.log} que registra todos los pedidos servidos por el proxy.
    El otro es \textit{debug.log}, que registra las acciones tomadas por el proxy.

    Esta configuracion puede ser cambiada editando \textit{src/main/resources/log4j.properties}.
    Recomendamos desactivar \textit{debug.log} a la hora de hacer benchmarking.
    Si bien no tiene un impacto importante en la perfomance, en el evento de que \textit{log4j} rote \textit{debug.log} se ve una caida temporal en la perfomance.

    \subsection{Ejecutar \duta}
    Una vez que se tiene la configuracion de log deseada, y se genera el binario con \awesome{Maven}, estamos listo para ejecutar \duta.
    Simplemente ejectuar \texttt{java -jar target/dUta-1.0-SNAPSHOT-jar-with-dependencies.jar} inicia el proxy con la configuracion por defecto.

    \subsubsection{Argumentos}
    Opcionalmente, se pueden pasar argumentos por linea de comandos que cambian los valores de configuracion por defecto.
    \begin{verbatim}
        --chain=ip:port    Chain to another proxy
        --port=port        Listen for requests on `port`
        --admin-port=port  Listen for admin requests on `port`
    \end{verbatim}

    A motivo de ejemplo, mostramos como ejecutar \duta para que se conecte a un segundo proxy corriendo en la misma pc.
    Asumamos que el segundo proxy ya esta utilizando el puerto 9999.
    Por lo tanto, hay que proveer a \duta de un puerto libre para escuchar, en este caso usaremos el 8888.

    Para ejecutar con esta configuracion corremos (saltos de linea introducidos por legibilidad):
    \begin{verbatim}
        java
            -jar target/dUta-1.0-SNAPSHOT-jar-with-dependencies.jar
            --port=8888
            --chain=localhost:9999
    \end{verbatim}

    \subsection{Administracion}
    Una vez que el servidor esta corriendo, se puede utilizar el protocolo de configuracion.

    Por ejemplo, para agregar un filtro que bloquee todos los requests hechos desde \textit{localhost}, podemos ejecutar:
    \begin{verbatim}
        curl
            -vvv
            -d '{"type": "deny-all", "apply": [{"ip": "127.0.0.1"}]}'
            -H "Content-Type: application/json"
            --basic --user "heman:masterofuniverse"
            localhost:1337/filters
    \end{verbatim}


% }}}

\end{document}
